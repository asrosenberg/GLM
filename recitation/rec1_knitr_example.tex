\documentclass[12pt]{article}\usepackage[]{graphicx}\usepackage[]{color}
%% maxwidth is the original width if it is less than linewidth
%% otherwise use linewidth (to make sure the graphics do not exceed the margin)
\makeatletter
\def\maxwidth{ %
  \ifdim\Gin@nat@width>\linewidth
    \linewidth
  \else
    \Gin@nat@width
  \fi
}
\makeatother

\definecolor{fgcolor}{rgb}{0.345, 0.345, 0.345}
\newcommand{\hlnum}[1]{\textcolor[rgb]{0.686,0.059,0.569}{#1}}%
\newcommand{\hlstr}[1]{\textcolor[rgb]{0.192,0.494,0.8}{#1}}%
\newcommand{\hlcom}[1]{\textcolor[rgb]{0.678,0.584,0.686}{\textit{#1}}}%
\newcommand{\hlopt}[1]{\textcolor[rgb]{0,0,0}{#1}}%
\newcommand{\hlstd}[1]{\textcolor[rgb]{0.345,0.345,0.345}{#1}}%
\newcommand{\hlkwa}[1]{\textcolor[rgb]{0.161,0.373,0.58}{\textbf{#1}}}%
\newcommand{\hlkwb}[1]{\textcolor[rgb]{0.69,0.353,0.396}{#1}}%
\newcommand{\hlkwc}[1]{\textcolor[rgb]{0.333,0.667,0.333}{#1}}%
\newcommand{\hlkwd}[1]{\textcolor[rgb]{0.737,0.353,0.396}{\textbf{#1}}}%
\let\hlipl\hlkwb

\usepackage{framed}
\makeatletter
\newenvironment{kframe}{%
 \def\at@end@of@kframe{}%
 \ifinner\ifhmode%
  \def\at@end@of@kframe{\end{minipage}}%
  \begin{minipage}{\columnwidth}%
 \fi\fi%
 \def\FrameCommand##1{\hskip\@totalleftmargin \hskip-\fboxsep
 \colorbox{shadecolor}{##1}\hskip-\fboxsep
     % There is no \\@totalrightmargin, so:
     \hskip-\linewidth \hskip-\@totalleftmargin \hskip\columnwidth}%
 \MakeFramed {\advance\hsize-\width
   \@totalleftmargin\z@ \linewidth\hsize
   \@setminipage}}%
 {\par\unskip\endMakeFramed%
 \at@end@of@kframe}
\makeatother

\definecolor{shadecolor}{rgb}{.97, .97, .97}
\definecolor{messagecolor}{rgb}{0, 0, 0}
\definecolor{warningcolor}{rgb}{1, 0, 1}
\definecolor{errorcolor}{rgb}{1, 0, 0}
\newenvironment{knitrout}{}{} % an empty environment to be redefined in TeX

\usepackage{alltt}
\usepackage[T1]{fontenc}
\usepackage{scrextend}
\usepackage{amsfonts, amsmath, amsthm, amssymb}
\usepackage[top = 1in, left = 1in, right = 1in, bottom = 1in]{geometry}
\usepackage{setspace}
\usepackage[bookmarks = true, hidelinks]{hyperref}
%\usepackage{breakurl}
\usepackage{bookmark}
\usepackage{booktabs}
\usepackage[flushleft]{threeparttable}
\usepackage{longtable}
\usepackage{rotating}
\usepackage{array}
\usepackage{dcolumn}
\usepackage{float}
\usepackage{subcaption}
\usepackage{siunitx}
\usepackage{natbib}
\usepackage[compact]{titlesec}
\usepackage{listings}
\usepackage{color}
\usepackage{baskervald}

\makeatletter

\makeatother

\pagestyle{plain}
\IfFileExists{upquote.sty}{\usepackage{upquote}}{}
\begin{document}


\title{Gulliver Unbound?}

\author{Andrew S. Rosenberg\thanks{Ohio State University}}

\maketitle

\begin{abstract} \noindent
What a AMAZING abstract!
\end{abstract}

\thispagestyle{empty}

\newpage
\setcounter{page}{1}
\doublespacing

\section*{Introduction}
I'm going to show the world how WRONG Fearon and Laitin are! Science!

\section*{Replication Study}
In this section, I will show that I can replicate the findings.



\begin{table}
\begin{center}
\begin{tabular}{l c }
\hline
 & Model 1 \\
\hline
(Intercept)    & $-6.666^{***}$ \\
               & $(0.739)$      \\
warl           & $-0.924^{**}$  \\
               & $(0.314)$      \\
gdpenl         & $-0.347^{***}$ \\
               & $(0.072)$      \\
lpopl1         & $0.257^{***}$  \\
               & $(0.073)$      \\
lmtnest        & $0.221^{**}$   \\
               & $(0.085)$      \\
ncontig        & $0.392$        \\
               & $(0.277)$      \\
Oil            & $0.886^{**}$   \\
               & $(0.279)$      \\
nwstate        & $1.717^{***}$  \\
               & $(0.339)$      \\
instab         & $0.625^{**}$   \\
               & $(0.236)$      \\
polity2l       & $0.024$        \\
               & $(0.017)$      \\
ethfrac        & $0.144$        \\
               & $(0.375)$      \\
relfrac        & $0.285$        \\
               & $(0.511)$      \\
\hline
AIC            & 978.436        \\
Log Likelihood & -477.218       \\
Num. obs.      & 6326           \\
\hline
\multicolumn{2}{l}{\scriptsize{$^{***}p<0.001$, $^{**}p<0.01$, $^*p<0.05$}}
\end{tabular}
\caption{Replication of Model 1 of Fearon and Laitin (2003).}
\label{table:coefficients}
\end{center}
\end{table}


Markdown/knitr is \emph{super} nice because it allows you to ensure that the
entire paper compiles properly---the results are the results. For example,
let's say I really must share with the reader that having oil leads to a
0.886 increase in the log odds of going to war. This feature is pretty slick because if you need to change the analysis for any reason, the text \emph{automatically} updates! From the perspective of the consumer, this feature is nice because you know that the results in the paper were generated by the analysis cited. It's open science!


\end{document}
